\documentclass[letterpaper]{article}

\usepackage[english]{babel}

% List styles

%Top Level
\renewcommand\theenumi{\Roman{enumi}}
\renewcommand\labelenumi{\theenumi.}

%Second Level
\renewcommand\theenumii{\arabic{enumii}}
\renewcommand\labelenumii{\theenumii.}

%Third Level
\renewcommand\theenumiii{\arabic{enumiii}}
\renewcommand\labelenumiii{\theenumiii)}

%Fourth Level
\renewcommand\theenumiv{\arabic{enumiv}}
\renewcommand\labelenumiv{\theenumiv.}

% Page layout (geometry)
\setlength\voffset{-1in}
\setlength\hoffset{-1in}
\setlength\topmargin{1in}
\setlength\oddsidemargin{1in}
\setlength\textheight{9.0in}
\setlength\textwidth{6.5in}
\setlength\footskip{0.0cm}
\setlength\headheight{0cm}
\setlength\headsep{0cm}

% Footnote rule
% TODO: Human Format this.
\setlength{\skip\footins}{0.0469in}
\renewcommand\footnoterule{\vspace*{-0.0071in}\setlength\leftskip{0pt}\setlength\rightskip{0pt plus 1fil}\noindent\textcolor{black}{\rule{0.25\columnwidth}{0.0071in}}\vspace*{0.0398in}}

% Pages styles
\makeatletter
\newcommand\ps@Standard{
  \renewcommand\@oddhead{}
  \renewcommand\@evenhead{}
  \renewcommand\@oddfoot{}
  \renewcommand\@evenfoot{}
  \renewcommand\thepage{\arabic{page}}
}
\makeatother
\pagestyle{Standard}

\title{The Constitution of Simmons Hall}
\date{\today} % Ensure that the constitution is dated whenever it was last compiled.

\begin{document}

\maketitle

\begin{enumerate}
\item PREAMBLE\setcounter{page}{1}\pagestyle{Standard}
\end{enumerate}

For decades at MIT, dormitory communities have been developing sophisticated systems of self-governance. These systems are entirely student-built and student-run, and they have received very little support or interference from the MIT administration. Despite or because of this (you be the judge), these governance systems have developed quite well.

According to the MIT administration, Simmons Hall is intended to signal a change from the current standards of dorm life. This is largely true. However, the governance of Simmons Hall will not be a change away from the self-governance model, but formalization and strengthening of it. The Simmons Hall governance model is intended to allow the dorm to have more independence and a stronger voice in the decisions that affect it. Moreover, it is intended to better define the relationship of Housemasters and Graduate Resident Tutors to the governance process, encouraging their participation and support but ensuring that they do not assume control over governance functions.

The Simmons Hall governance model is also founded on the concept of the House as an entity that is more than just a building and more than simply the sum of its residents. All residents are supported by the House physically and socially. Reciprocally, the House requires support from all of its residents. This House model works on several basic principles:

Principle 1: All residents have an implicit responsibility to the House. There is not a select group of people who handle all the functions of House governance and whom you can blame if dorm life is not what you want it to be. It is up to everyone. If you want Simmons Hall to be a good place, you have to get involved and make it that way. If you think Simmons Hall sucks, then there's no one you can blame but yourself.

Principle 2: There are varying levels of responsibility to choose from. You can run for an elected Office, or volunteer to serve on a committee, or show up to House meetings, or on the most basic level, have respect for the House and for your fellow residents.

Principle 3: The House speaks for the House. When the House makes a decision, it is explicit that the decision was made through a democratic process that involves all House residents. There are no representatives who ``call the shots'' for everyone.

Principle 4: If you do not like something, change it. While traditions are important to building culture, the House should not keep doing something simply because people did it the year before, and other people did it the year before that. Times keep changing and the House should keep thinking creatively and coming up with new ways to improve House life.

Principle 5: House governance should be part of the educational experience for all residents. Students and faculty alike should feel that they learned something from being an active member of the House. They should also feel that they had some degree of fun and that their educational experience was better because they had a release from the rigor of MIT academics and research.

This is the philosophy behind the governance of Simmons Hall.
\\
\\
Jeff Roberts\newline
Simmons Hall Founders Group, 1999-2002\newline
Governance Subcommittee Chair\newline
September 3, 2002
\clearpage

\begin{enumerate}
\setcounter{enumi}{1} % Make sure that we account for the preamble.
\item GENERAL DEFINITIONS

\begin{enumerate}

\item Constitution of Simmons Hall

\begin{enumerate}

\item The purpose of the Constitution is: (a) to establish the overall model for governance of the House; (b) to define the roles of some of the individuals and groups who are involved in the governance of the House; (c) to define some of the procedures that guide the proper functioning of House governance.

\end{enumerate}

\item The House

\begin{enumerate}

\item The House refers to the collective residents and communities residing within the Simmons Hall dormitory of the Massachusetts Institute of Technology.

\end{enumerate}

\item House and Chair Policies

\begin{enumerate}

\item House and Chair Policies supplement the Constitution in guiding House governance, and may be created and revised continually over time.

\item House Policies may include, but are not limited to, policies on the use of House facilities and House funds, rules of conduct such as quiet hours or regulations on the use of alcohol, policies that establish the roles and responsibilities of new House governance groups or positions, and policies that define or clarify House governance procedures.

\item Chair Policies are determined by each current chair. These policies serve to clarify chairs' roles and responsibilities as assigned by the constitution. They also serve to inform members of the House of officers roles and responsibilities. The content of each chair's policies can be changed by said chair, however these changes must be announced at a House Meeting to take effect. It is encouraged that a chair solicit house opinion before changing his or her policies.

\end{enumerate}

\end{enumerate}

\item THE HOUSE

\begin{enumerate}

\item Introduction

\begin{enumerate}

\item The House makes many decisions that impact all its residents. It raises funds to use towards House-shared resources and events. The governance of Simmons Hall is established on the principle that when an issue affects all residents, every resident should have a voice in making a decision on that issue.

\end{enumerate}

\item House Membership

\begin{enumerate}

\item Every official resident of Simmons Hall is a member of the House. Each member may cast one vote in any House decision.

\end{enumerate}

\item Role and Responsibilities of the House

\begin{enumerate}

\item The House, by collective action, makes all major house-wide decisions and expresses the official opinion of Simmons Hall on any matter of interest to the residents of Simmons Hall.

\item The House is responsible for: (a) establishing House and Chair Policies, and making changes to standing House Policies and the Constitution; (b) allocating House funds to be used towards House-shared resources and events; (c) discussing and deciding upon any issue of relevance to Simmons Hall or to MIT at large; (d) approving any resolution expressing the official opinion of the residents of Simmons Hall on any such issue (e) electing House Officers to carry out the will of the House.

\end{enumerate}

\item House Procedures

\begin{enumerate}

\item All official business of the House is discussed at meetings that are open and announced to all members of the House and occur at least once within every two-week period during spring and fall semesters that MIT classes are in session. House Officer meetings shall be held at least a half an hour before general House meetings.

\item Any House member may propose an item to be discussed or decided upon by the House by submitting a proposal to the Chair of the House, who must place that item on the agenda of the next House meeting if the agenda for that meeting has not been set.

\item For all proposals brought before the House, four actions are available to the House: Approve, Reject, Table, and Move-to-Full-Forum. The Chair of the House should remind participants at the start of each semester that these options are available. Voting may take place during meetings of the House or by ballot in the case of Move-to-Full-Forum. No item may be voted upon until after it has been discussed at a House meeting.

\item At least one author of the proposal must be present to advocate for it. Should no author be present, the proposal shall be Tabled automatically, unless an author has presented the House Chair an alternative advocate for the proposal in advance. Would the same proposal be automatically tabled in this manner for the third consecutive time, it will instead be automatically rejected to avoid cluttering the agenda with orphaned proposals.

\item During debate on a proposal, motions may be raised. After a member of the House states intention to raise a motion, another member must second the motion in order for it to have any effect (i.e. ``be raised'').

\item Proposals may be approved during a House meeting following a motion to vote. Once seconded, the House chair should direct and tally a vote, unless a motion to object is raised and seconded. In such cases, the House should return to debate on the proposal. Proposals are approved by a majority vote yes, with the exception that proposals for Constitutional Amendments require a two-thirds majority vote for approval.

\item In the case where those assembled feel they are unanimously in favor of approving the proposal, a motion to white-ballot can be used. When a motion to white-ballot is raised, the House chair should ask those assembled if any are opposed. If any member of the House present is opposed, then the motion fails and the House returns to Debate. Once a motion to white-ballot fails for a particular measure, no further motions to white-ballot may be raised for that measure. A motion to white-ballot may not be used to approve a constitutional amendment.

\item A motion to force-vote may be used to counter the case when a vote cannot be held due to frivolous objections. Once raised in the course of normal debate, a motion to force-vote causes the House Chair to poll the support for a vote of those assembled. This vote cannot be interrupted by a motion to object or any other motion. If the force-vote passes with majority support then the meeting should proceed as if a motion to vote had been raised, with the exception that no motions, including motions to object, may interrupt the vote. For each proposal, no more than one motion to force-vote may be raised in any five minute period.

\item Proposals are moved-to-full-forum by a greater than one-third vote yes. A motion to move-to-full-forum takes precedence over a motion to approve but a motion to object can still return the House to debate. When a proposal is moved-to-full-forum, the Secretary should create a description of the issue of contention with input from one or more proponents and opponents. The Secretary would then publicly post the description and issue a full house ballot on the issue to be concluded before the next house meeting. The results of this vote would be final.

\item A motion to force-move-to-full-forum works like to a motion to force-vote with two exceptions. First it requires only one-third support to proceed and second an uninterrupted motion to move-to-full-forum, not a motion to vote, is the result. For each proposal, no more than one such motion may be raised in any five minute period.

\item Proposals are tabled by a majority vote. A motion to table takes precedence over a motion to move-to-full-forum and a motion to approve but a motion to object can still return the House to debate. When a proposal is tabled, it is moved to the next house meeting.

\item A motion to force-table entails an immediate and uninterrupted vote on house support for said motion. Should a majority agree with the motion, then an uninterrupted motion to table should proceed. For each proposal, no more than one such motion may be raised in any five minute period.

\item For all the motions above, the fraction of votes required is to be taken as a fraction of those electing to vote on the motion. Any members of the House who are not present or who choose to abstain shall not be included in the count of the total number of votes.

\end{enumerate}

\item House Executive Positions

\begin{enumerate}

\item The House elects one individual to serve as Chair of the House, one individual to serve as Secretary, one individual to serve as Treasurer, and one individual to serve as President, each for a one year term.

\item The Chair of the House is the chief administrator of the House's internal affairs. The Chair is responsible for: (a) creating and distributing to all members a schedule of House meetings for each academic semester, prior to the start of that semester; (b) explaining to the House how House meetings work at the start of each semester; (c) adding proposals to the agenda of the upcoming House meeting; (d) creating and distributing the agenda for each House meeting at least one day prior to the start of that meeting; (e) facilitating and maintaining order at all meetings of the House, following an established set of procedures and conducting meetings in a fair way that allows all members an opportunity to speak; (f) conducting elections; (g) generally overseeing the business of the House and directly overseeing the Historian, Social Chair, Facilities Chair, Technology Chair and the Rooming Chairs.

\item The President is the chief diplomat of Simmons Hall and the administrator of the House's external affairs. The President is responsible for: (a) being the primary representative of the House to any outside groups, including, but not limited to, the Simmons' caterer, other dormitories and living groups, student governments, other student groups, MIT administration, MIT faculty, and groups outside MIT; (b) presenting the official opinion of the House, as determined by the House, to the aforementioned groups; (c) advocating and negotiating on behalf of Simmons Hall to the aforementioned groups; (d) presenting items for discussion and decision by the House on behalf of the aforementioned outside groups; (e) representing Simmons Hall to the Dormitory Council and casting votes on behalf of Simmons Hall; (f) communicating regularly with the Housemasters; (g) reporting regularly to the House; (h) inviting outside guests to address the House; (i) directly overseeing the FROSH Chairs, Athletics Chair, and Publicity Chair.

\item The President shall have the power to convene Ad-Hoc Committees to assist him or her in the execution of the responsibilities of the office. Should the President believe it necessary, he/she has the power to form an Ad-Hoc Judicial Committee. The Chair of these committees, if not the President himself/herself must be approved by a majority vote of the House. Should the President be accused, the Chair of the House has the power to convene an Ad-Hoc Judicial Committee.

\item The Secretary is responsible for: (a) documenting all business of the House; (b) keeping a public record of all decisions made by the House; (c) maintaining an archive to ensure that such documentation is available to all present and future House members; (d) maintaining a public record of the events and discussions that happen in the House and presenting those records at every house meeting; (e) running full forum votes by publicly posting a description of the measure(s) being voted on, then issuing a ballot on said measures to be concluded before the next house meeting; (f) serving in lieu of the Chair of the House if the Chair is temporarily unable to serve.

\item The Treasurer manages the finances of the House and implements the financial decisions made by the House. The Treasurer is responsible for: (a) receiving signatory privileges for all House bank accounts, including the student activity account maintained by MIT; (b) keeping record of all financial decisions made by the House; (c) reimbursing individuals for House-related expenses approved by the House; (d) reporting at every meeting of the House, presenting the balances of all
House bank accounts and an itemized report of all account activity, and making its reports available for viewing by all House Members; (e) Treasurer's report should include: current balances, anticipated budgets, and effects of current proposals on the budget.

\item If the Chair position becomes vacant, the President assumes the responsibilities of the Chair until a special election can be held to fill the position. Similarly, if the Presidency becomes vacant, the Chair assumes the responsibilities of the President until a special election can be held. In both cases, the special election must be announced immediately and run in compliance with the procedures for regularly scheduled elections.

\end{enumerate}

\end{enumerate}

\item COMMITTEE CHAIRS OF THE HOUSE

\begin{enumerate}

\item Committee Definitions and Membership

\begin{enumerate}

\item A committee is a group of House members who collectively focus on particular sets of issues of importance to Simmons Hall. Committees may be established on a Standing or Ad-Hoc basis.

\item An Ad-Hoc Committee for any purpose may be convened by the President or by virtue of a majority vote of the House.

\item Any Standing Committee must be formed by action of the House in the form of an amendment to this Constitution and included in the following section.

\item There are fourteen Committee Chairs leading the Standing Committees. They are: (a) the Social Chair; (b) the Technology Chair; (c) the Facilities Chair; (d) the Electrical Engineering Lab Chair; (e) the Library Chair; (f) the Entertainment Chair; (g) the Reservations Chair; (h) the Workshop Chair; (i) the Rooming Chairs; (j) the FROSH Chairs; (k) the Athletics Chair; (l) the Publicity Chair; (m) the Historian; (n) the Kitchen Chair. These Chairs and their committees must exist at all times, and are elected for a one year term. The roles and responsibilities of these committees are stated in this Article of the Simmons Hall Constitution.

\item Each Committee Chair is responsible for: (a) maintaining a committee membership list; (b) scheduling and facilitating meetings of the committee; (c) reporting on the activities of the committee at House meetings. If the Chair(s) fail to report at House meetings, the Chair of the House may choose either: (a) to allow the remaining committee members to elect a new Chair; (b) to appoint a new Chair; (c) to dissolve the committee.

\item The House elects an individual to serve one-year terms as Chair of a Committee of the House defined in this Article. These Chairs are considered House Officers.

\item Membership on any committee must be open to all members of the House. A member of the House may join a committee by contacting the Committee Chair or the Chair of the House and asking to be added to the membership list.

\item If a committee member fails to report to meetings or otherwise fails to carry out the responsibilities required of the individual by that committee, the Committee Chair may choose to remove that individual from the membership of the committee.

\end{enumerate}

\item Committee Procedures

\begin{enumerate}

\item All business of each committee is discussed at meetings that are open and announced to all members of the committee, and preferably announced to all Members of the House. Committees may encourage discussion outside of meetings; however, all modes of discussion must allow for participation from every member of the committee.

\item Committee meetings should be held at appropriate times given the particular topic of the committee's attention. Committee meetings should typically occur once each month, unless there is no business to be discussed at that time.

\item Committees make decisions by group consensus or by majority vote. No decisions are made until after the full membership of the Committee is given a chance to comment and/or vote. Committee decisions are reported to the House by the Committee Chair. If any member of a committee disputes a decision reported by the Committee Chair, the Chair of the House may call a vote of the committee membership.

\end{enumerate}

\item Role and Responsibilities of the Social Chair

\begin{enumerate}

\item The Social Chair leads the Social Committee comprised of all lounge representatives.

\item The Social Chair is responsible for regularly reporting to the House Chair and the House.

\item The Social Committee is responsible for organizing several House wide events each term.

\end{enumerate}

\item Role and Responsibilities of the Technology Chair

\begin{enumerate}

\item The Technology Chair leads the Technology Committee that maintains the computing resources managed by the House.

\item The Technology Chair is responsible for (a) organizing, scheduling and running meetings of the Technology Committee; (b) ensuring that the Technology Committee accomplishes its goals; (c) reporting to the House Chair; (d) reporting regularly to the House.

\item The Technology Committee is responsible for: (a) debugging and updating the Simmons Database System, House mailing lists, and House web server; (b) proposing, implementing, and maintaining technological solutions to meet and improve other House computing needs.

\item The Technology Committee must maintain high standards of confidentiality and ethics when dealing with the Simmons database and sensitive issues.

\end{enumerate}

\item Role and Responsibilities of the Facilities Chair

\begin{enumerate}

\item The Facilities Chair leads the Facilities Committee comprised of the Electrical Engineering Lab Chair, the Library Chair, the Entertainment Chair, the Reservations Chair, the Workshop Chair, and the Kitchen Chair.

\item The Facilities Chair is responsible for (a) organizing, scheduling and running meetings of the Facilities Committee; (b) ensuring that the Electrical Engineering Lab Chair, the Library Chair, the Entertainment Chair, the Reservations Chair, the Workshop Chair, and the Kitchen Chair fulfill their jobs to their fullest extent; (c) working with the House Manager to accomplish tasks; (d) reporting to the House Chair; (e) reporting regularly to the House.

\item The Facilities Committee is responsible for creating, recommending, and overseeing the implementation of policies regarding the use of House-owned equipment and spaces.

\end{enumerate}

\item Role and Responsibilities of the Electrical Engineering Lab Chair

\begin{enumerate}

\item The Electrical Engineering Lab Chair is part of the Facilities Committee and maintains the lab facility on the seventh floor. The Electrical Engineering Lab Chair is responsible for (a) organizing training sessions for residents who want to use the lab (b) ensuring that residents are properly trained and certified to use the lab (c) reporting to the Facilities Chair; (d) reporting regularly to the House.

\end{enumerate}

\item Role and Responsibilities of the Library Chair

\begin{enumerate}

\item The Library Chair is part of the Facilities Committee and supervises the library on the second floor. The Library Chair is responsible for (a) creating and maintaining a checkout system for library materials; (b) organizing the library; (c) obtaining new books when the desire is expressed by the House; (d) reporting to the Facilities Chair; (e) reporting regularly to the House.

\end{enumerate}

\item Role and Responsibilities of the Entertainment Chair

\begin{enumerate}

\item The Entertainment Chair is part of the Facilities Committee and maintains the Simmons Hall movie, video game, and board game collection.

\item The Entertainment Chair is responsible for: (a) maintaining an easily accessible, public list of movies, video games, and board games (b) soliciting opinions from the House regarding what new movies, video games, board games, or Pay-Per-View programming should be purchased; (c) requesting funding for the purchase of new movies, video games, board games and Pay-Per-View programming; (d) purchasing new movies, video games, and board games to be added to the Simmons Hall movie collection; (e) managing the Pay-Per-View account; (f) reporting to the Facilities Chair; (g) reporting regularly to the House.

\end{enumerate}

\item Role and Responsibilities of the Reservations Chair

\begin{enumerate}

\item The Reservations Chair is part of the Facilities Committee and oversees the reservation process of all Simmons public spaces according to the Policies of the Reservations Chair. The Reservations Chair is responsible for reporting regularly to the Facilities Chair and the House.

\end{enumerate}

\item Role and Responsibilities of the Workshop Chair

\begin{enumerate}

\item The Workshop Chair is part of the Facilities Committee and maintains the workshop facility in the basement. The Workshop Chair must communicate with the administrators of the MIT Environment, Health, and Safety Headquarters to ensure that residents are properly trained and certified to use the workshop.

\item The Workshop Chair is responsible for: (a) organizing training sessions for residents who want to use the workshop; (b) communicating with the House Manager to grant card access to the workshop for residents who have been properly trained and certified; (c) reporting to the Facilities Chair; (d) reporting regularly to the House.

\end{enumerate}

\item Roles and Responsibilities of the Rooming Chairs

\begin{enumerate}

\item The Rooming Chairs manage room assignments for new and continuing residents according to the Rooming Policies.

\item There are always two Rooming Chairs. At each election, the House elects an individual to serve as Rooming Chair for a two year term.

\item The Rooming Chairs are responsible for: (a) managing and executing the Rooming Policies; (b) maintaining a record of all Simmons Hall residents and their assigned rooms; (c) communicating regularly with the Housemasters, House Manager and Desk Captain; (d) communicating with the MIT Housing Office and DormCon Housing Committee on housing-related issues, with the assistance of the President; (e) providing assistance to the Desk Captain upon request whenever residents are moving into or out of the building; (f) reporting regularly to the House Chair and the House.

\end{enumerate}

\item Role and Responsibilities of the Freshmen Recruitment Organizers for Simmons Hall (FROSH) Chairs

\begin{enumerate}

\item At each election, the House will elect two individuals to serve as FROSH Chairs for one-year terms.

\item The FROSH Chairs lead the FROSH Committee in overseeing all activities related to the housing selection process for incoming new students.

\item The FROSH Chairs are responsible for (a) organizing, scheduling and running meetings of the FROSH Committee; (b) ensuring that the FROSH Committee accomplishes its goals; (c) reporting to the President; (e) reporting regularly to the House. \item The FROSH Committee is responsible for: (a) working with the Publicity Chair to produce material for the Interactive Introduction to the Institute (I3) video, the Guide to First-Year Residences, and any
other material sent to incoming students regarding residential life; (b) overseeing the planning and organization of events during Campus Preview Weekend and Orientation Week for prospective or incoming students; (c) providing assistance to the Desk Captain during any times at which residents are moving into or out of the building; (d) communicating with the DormCon Residence Orientation Chair, I3 staff, and administration on any of the above issues, with the assistance of the President.

\item The FROSH Committee recruits specific volunteers to undertake particular activities, such as the production of the I3 video, or organizing a Campus Preview Weekend party. These volunteers are automatically and necessarily considered members of the FROSH Committee.

\item The FROSH Committee may request that its members be granted early returns from DormCon for Orientation Week. Only members of the FROSH Committee who assist in the organization or implementation of Orientation week activities may be granted early returns.

\end{enumerate}

\item Role and Responsibilities of the Athletics Chair

\begin{enumerate}

\item The Athletics Chair heads the Athletic Committee made up of the Simmons' Intramural team captains. The Athletics Chair shall be responsible for (a) organizing and coordinating House participation in external athletic programs, especially the MIT Intramural Sports Program (b) reporting to the President; (c) reporting regularly to the House.

\item The Athletics Chair should communicate with the administrators of the MIT Intramural Sports Program and take steps to coordinate Simmons' involvement including, but not limited to, surveying interest in participating in a given sport, and helping to recruit members for intramural teams. The Athletics Chair should report regularly to the House regarding the status of the House's participation in Intramural Sports.

\item The Athletics Chair should be responsible for bringing forth all proposals to the House regarding dues for team participation and proposals for equipment for Simmons' Intramural teams.

\item The Athletics Chair will also have an oversight role over the various Intramural teams and should take steps to ensure that teams remain active in their league and do not incur punitive fines. The Athletics Chair can recommend to the House not to fund historically negligent teams.

\end{enumerate}

\item Role and Responsibilities of Publicity Chair

\begin{enumerate}

\item The Publicity Chair is tasked with the promotion and publicity of the dorm and its events through the use of media content including, but not limited to, posters, flyers, web campaigns, 7K display, and the I3 video. The Publicity Chair is responsible for (a) promoting the dorm; (b) assisting the House Team and House Officers in publicizing dorm events; (c) writing and releasing an alumni newsletter as a high-level overview of occurrences at Simmons; (d) reporting to the President; (e) reporting regularly to the House.

\end{enumerate}

\item Role and Responsibilities of the Historian

\begin{enumerate}

\item The Historian is tasked with the high-level goal of preserving a record of Simmons history for future students and alumni. The Historian is responsible for (a) ensuring the biweekly records written by the Secretary are complete; (b) assisting the Publicity Chair with the writing and publishing of the alumni newsletter; (c) creating a yearly retrospective as a high-level and student-opinion-aware summary of the year.

\item The Historian is responsible for ensuring that the history recorded by the Secretary, the Publicity Chair, and the Historian is as unbiased as possible and reflects student opinions.

\end{enumerate}

\item Roles and Responsibilities of the Kitchen Chair

\begin{enumerate}

\item The Kitchen Chair is part of the Facilities Committee and maintains the kitchen facilities in Simmons.

\item The Kitchen Chair is responsible for: (a) maintaining a stock of permanent cookware available to Simmons residents; (b) maintaining a stock of basic ingredients available to Simmons residents; (c) ensuring that residents keep kitchens clean.

\end{enumerate}

\end{enumerate}

\item OFFICER PROVISIONS

\begin{enumerate}

\item Elections

\begin{enumerate}

\item Only undergraduate members of the House are eligible to serve as officers. No individual may hold more than one office simultaneously, except in cases where an office remains unfilled for an extended length of time.

\item All elections must be open to all members of the House, whether or not they are able to attend House meetings.

\item The election process is overseen by the Elections Organizer. By default, the Elections Organizer is the Chair of the House. However, if the Chair intends to be a candidate in the election or otherwise wishes not to be Elections Organizer, the Chair must select another undergraduate member of the House (who must not intend to be a candidate) to be the Elections Organizer. This selection must be approved by a simple majority vote of the House. The selection of Elections Organizer must occur before the beginning of nominations. The Elections Organizer cannot be nominated to be a candidate. The Elections Organizer hears all complaints related to the fairness of the election procedure.

\item Nominations for officer positions will occur during the week preceding the elections House meeting. A nominee will be placed on the ballot if their name is mentioned. People can nominate themselves and nominate multiple times. One can be nominated by proxy. The Chair must nominate anyone who notified him/her by two days before the House meeting.

\item During the week before the announced election meeting and the final meeting of the term, there is a special election meeting. During this meeting, all candidates will be given three minutes to present their platform and one minute to answer questions. A further several minutes will be dedicated to questioning any/all of the candidates for a particular office. Proxies may read the platform of missing candidates, but may not take part in the question answering sessions.

\item New officers take office during the first meeting of the spring term. During the transition period (the time between election results and the first spring House meeting), outgoing officers have the responsibility of training the incoming officers and handing over turnover files. Outgoing officers also have the responsibility of
helping to run the first meeting of the spring term.

\item An office becomes vacant when the individual holding that office resigns or is no longer an official resident of Simmons Hall. An Officer may resign by informing the Chair of the House of one's resignation, or the Chair of the House may assume that an officer has resigned if that officer fails to communicate with the House or otherwise fails to undertake the responsibilities of that office.

\item If a House Officer position, other than one specified in article III.5.7, becomes vacant mid-term, the resigning officer should be reminded of his/her duty to train the replacement. The Chair of the House should notify the House that the spot has become vacant and that they may demonstrate interest within the week. At the end of the week, one of two outcomes has occurred: (a) 1 or 0 people have contacted the chair. In this case, the chair is permitted to appoint said person (if one is found) to the office in question. The Chair must ratify the selection at the next House meeting with a simple majority vote; (b) 2 or more people have contacted the chair. In this case, the chair (with the help of other officers) must administer a one-week mid-term election for the House to select between the candidates. Additionally, if the end of the one-week special election
would fall within one month of the end of a regular election, then the House Chair may exercise discretion in deciding whether to run a vote / appoint directly / wait for the main election.

\item The previous section, Article 5 Section 1 Item 8, does not apply to the instance in which the Secretary choses to assume the role of House Chair permanently.

\item Provisionally elected officers take office immediately upon election and fill the vacant office until the time of the next regular election.

\end{enumerate}

\item Impeachment

\begin{enumerate}

\item Any member or group of members of the House may propose an Article of Impeachment against an officer, if that member or group feels that: (a) the officer is neglecting the responsibilities of that office; (b) the officer is abusing the authority granted to that office; (c) the Officer is not acting in the interest of the House. A proposed Article of Impeachment must include an explanation that addresses at least one of these three points.

\item A proposed Article of Impeachment is placed on the House agenda, discussed, and voted upon in the same manner as is any other item of House business.

\item If the Chair of the House is the object of the Article of Impeachment, the Secretary of the House assumes the responsibilities of the Chair for all business relating to the Article of Impeachment.

\item If an Article of Impeachment is approved by a House vote, the Chair of the House holds a hearing open to all members of the House. The proponent of the Article of Impeachment, the object of the Article of Impeachment, and the Chair are given the opportunity to speak at this hearing, and may invite witnesses to speak on their behalf, but any witness may be dismissed by the Chair of the House on the grounds of irrelevance. No other individuals are allowed to speak at this hearing.

\item The Chair of the House may decide, based on the evidence presented at the hearing, to remove an individual from office if and only if: (a) the information presented in the Article of Impeachment is truthful in content; and (b) the Article of Impeachment sufficiently explains why the impeached individual should be removed from office. After deliberation, the Chair presents the decision, along with a written explanation, to the House.

\item An office that is unoccupied as the result of impeachment proceedings is then treated in the same manner as any other vacancy, except that the removed officer is no longer eligible to hold that office.

\end{enumerate}

\end{enumerate}

\item THE HOUSE SUPPORT TEAM

\begin{enumerate}

\item Live-In House Support Positions

\begin{enumerate}

\item MIT specially selects individuals to live in Simmons Hall in order to ensure the health, safety and well-being of residents and to promote a vibrant and diverse social life. These include Housemasters and Graduate Resident Tutors (GRTs), Visiting Scholars, and the Residential Life Area Director among others. Along with the House Manager and staff, these individuals collectively form a House Support Team, with the Housemasters serving as its head, that operates independently of
the House government but must remain in contact with the House government in order to ensure a mutually supportive working relationship.

\item Housemasters, GRTs, and any other individuals who are specially selected by MIT to live in Simmons Hall are all considered Members of the House, but do not have any official standing within the House, nor do they have any official responsibilities to the House. Nonetheless, they are encouraged to participate in all activities of the House, and to offer their support where the House would benefit from the specialized assistance they may be able to provide.

\end{enumerate}

\item The Housemaster Role

\begin{enumerate}

\item The Simmons Hall Housemaster team consists of the Housemasters and Associate Housemasters, who are faculty members living with their families in Simmons Hall. The Housemasters and Associate Housemasters are appointed to their positions by the Office of the Dean for Student Life. They are expected to integrate fully into Simmons and to support the entire dorm socially and intellectually.

\item The Housemaster role in Simmons falls into four areas: (a) They hold frequent social events, and participate actively in the social events held by House residents, in order to get to know as many undergraduates as possible; (b) they help residents deal with problems, ranging from inter-personal or inter-group issues to those of a highly personal and individual nature; (c) they supervise the Graduate Resident Tutors; (d) they report to the rest of the MIT faculty and to the MIT administration.

\item When a new Housemaster or Associate Housemaster must be selected, a committee of House Officers is formed to review candidates and make recommendations to the Dean for Student Life.

\end{enumerate}

\item The Graduate Resident Tutor Role

\begin{enumerate}

\item Graduate Resident Tutors are MIT graduate students hired by the Office of the Dean for Student Life to live in Simmons Hall, sometimes along with partners or families. They are expected to integrate fully into the dorm and to provide support to a specific set of residents living in a particular area of the building.

\item The GRT role in the House falls into four areas: (a) They serve as advisors and mentors to residents, particularly those living within their area; (b) they respond to crises and conflicts that occur within the dorm, relying on their training and judgment; (c) they serve as advocates for the interests of students, as liaisons between students and Housemasters, and sometimes as liaisons between students and the administration; (d) they serve as community facilitators, encouraging interaction and providing recreational opportunities for residents, particularly those living within their area.

\item The GRT role in the dorm expressly excludes any disciplinary component. Only the Housemasters may officially recommend disciplinary action against a resident of Simmons Hall.

\item When new GRTs must be hired to fill open positions in Simmons Hall, a House committee of House Officers and the Housemaster team is formed to review all candidates and provide a list of recommended candidates to the Office of the Dean for Student Life.

\end{enumerate}

\item Other Special Live-In Positions

\begin{enumerate}

\item Other individuals who are selected to live in Simmons Hall may include Visiting Scholars and their families, Residential Life Associates, or possibly others to be defined in the future.

\item Individuals specially selected by an MIT authority to live in Simmons Hall, while Members of the House, are not required to have any special responsibilities to the House. However, the House may establish House Policies regarding the roles these individuals are expected to play in the House. Live-in roles as defined by the House must complement, and must not contradict, the roles as defined by the MIT authority responsible for appointing or hiring the individuals.

\end{enumerate}

\item Non-Live-In House Support Positions

\begin{enumerate}

\item Individuals who play significant support roles within the Simmons Hall community but are not residents of Simmons Hall may include the House Manager, staff serving under the House Manager, the dining hall staff, and others. These individuals are not considered Members of the House, nor do they have any official standing with respect to the House, nor do they have any official responsibilities to the House. Nonetheless, the House, through its officers, is encouraged to
communicate them, and ask for their support in areas where the House would benefit from the specialized assistance they may be able to provide.

\end{enumerate}

\end{enumerate}

\item USE OF HOUSE RESOURCES

\begin{enumerate}

\item Use of House Funds

\begin{enumerate}

\item Every House resident has an equal opportunity to benefit from House funds.

\item All proposals made to the House should have an explicit time-frame built in. This is especially important so that treasurers do not need to maintain logs indefinitely of approved, allocated money that has never been spent. The ``default'' expiration of allocations (i.e. those that do not specify otherwise) will be at
the end of the term in which the money was intended to be spent, or at the end of the current term if no intended date is specified.

\item In order to carry out the duties of their offices, in accordance with Article II Section 5, the Chair of the House, the President, and the Social Chair shall be granted a fund of \$500 at his or her discretion. If spent, this fund may only be replenished by vote of the House Committee.

\item Simmons Hall is a member of the Dormitory Council (DormCon) and pays all associated taxes.

\end{enumerate}

\end{enumerate}

\item CONSTITUTIONAL VALIDITY

\begin{enumerate}

\item Constitution Ratification

\begin{enumerate}

\item This Constitution is offered by the Governance Subcommittee of the Simmons Hall Founders Group to the residents of Simmons Hall. The residents of Simmons Hall have made certain changes to reflect the first year of Simmons' existence. The residents have also chosen to ratify this Constitution with a two-thirds vote, following a consensus from the House that the voting method is appropriate.

\end{enumerate}

\end{enumerate}

\end{enumerate}

\end{document}
